% Chapter Template

\chapter{Results} % Main chapter title
\label{chap:results} % Change X to a consecutive number; for referencing this chapter elsewhere, use \ref{ChapterX}
This chapter will summarize the results of this work. First, the results of simulating the susceptible population
in the region of Hesse will be presented. Then the results of a sensitivity analysis for the variables $\alpha$ and
$q$ will be shown. Specifically the sensitivity analysis will later be used in \hyperref[chap:discussion]{Chapter
\ref*{chap:discussion} - Discussion} to explain the results of the simulations.


%----------------------------------------------------------------------------------------
%	SECTION 1
%----------------------------------------------------------------------------------------

\section{Simulating the susceptible population of Hesse}
During this work we simulated the susceptible population of Hesse. 26 regions were simulated over a time period of
76, 60 and 50 days respectively. We will present both the absolute and percentage difference between the simulated
and the original data for each time frame.

\textcolor{red}{redo box plots with x-axis description}

%-----------------------------------
%	SUBSECTION 1
%-----------------------------------
\subsection{Simulating susceptibles in a 76 day time frame}
We first wanted to observe how well the simulation performs in a time frame of 76 days. Since the number of susceptible
individuals is much greater then any other group at any given data point, changes in this group can be difficult to
observe. Because of this we decided to calculate and compare the number of individuals that migrated from the susceptible
to the exposed group instead. This was done by subtracting the number of susceptible individuals at data point \I{t=x} from
the start point \I{t=0}. The result is the total change of susceptibles at any given time point, which is equivalent to the 
sum of all exposed individuals at any given time point. These results are much easier to compare and understand.
\hyperref[fig:76_sim_expl]{figure \ref*{fig:76_sim_expl}} shows three graphs that illustrate this process.


\begin{figure}
	\centering
	\begin{subfigure}[b]{0.3\textwidth}
		\centering
		\includegraphics[width=\textwidth]{./figures/76d/24_Darmstadt-Dieburg.png}	
		\caption{}
	\end{subfigure}
	\hfill
	\begin{subfigure}[b]{0.3\textwidth}
		\centering
		\includegraphics[width=\textwidth]{./figures/76d/13_Main-Kinzig-Kreis.png}	
		\caption{}
	\end{subfigure}
	\hfill
	\begin{subfigure}[b]{0.3\textwidth}
		\centering
		\includegraphics[width=\textwidth]{./figures/76d/10_Limburg-Weilburg.png}	
		\caption{}
	\end{subfigure}
	\caption{Three exemplary results of a simulation of the exposed individuals.
		The original (``orig'') data is drawn in blue and the simulated (``sim'') data is drawn in orange.
		The number of simulated exposed individuals can be greater (image (A), region ``Darmstadt Dieburg''), smaller 
		(image (B), region ``Mein Kinzig Kreis'') or about the same (image (C), region ``Limburg Weilburg''), as 
		the originally observed number of exposed.
		}
	\label{fig:76_sim_expl}
\end{figure}

Furthermore we analyzed the percentage deviation of original and simulated data for each time step in each region. The results
are shown in a box plot in \hyperref[fig:76_sim_box]{figure \ref*{fig:76_sim_box}}. The three regions ``Werra-Meissner-Kreis'',
``Marburg-Biedenkopf'' and ``Limburg-Weilburg'' are listed separately in figure \ref*{fig:76_sim_box}, in order to make the scales
more readable.


\begin{figure}
	\centering
	\begin{subfigure}[b]{0.4\textwidth}
		\centering
		\includegraphics[width=\textwidth]{./figures/76d/deviation_box76_alt1.png}	
	\end{subfigure}
	\begin{subfigure}[b]{0.4\textwidth}
		\centering
		\includegraphics[width=\textwidth]{./figures/76d/deviation_box76_alt2.png}	
	\end{subfigure}
	\begin{subfigure}[b]{0.4\textwidth}
		\centering
		\includegraphics[width=\textwidth]{./figures/76d/deviation_box76_alt3.png}	
	\end{subfigure}
	\begin{subfigure}[b]{0.4\textwidth}
		\centering
		\includegraphics[width=\textwidth]{./figures/76d/deviation_box76_alt4.png}	
	\end{subfigure}
	\caption{Shown are box plots of the percentage deviation, of every simulated region relative to the original data.}
	\label{fig:76_sim_box}
\end{figure}

\textcolor{red}{rephrase for deviation factor and meaning}
Figure \ref*{fig:76_sim_box} shows that the simulated regions have a wide range of median values regarding the deviation.
The numbers are ranging between a median deviation factor of about -0.8 and +0.75. This means, that the model calculated
between 80 percent less or 75 percent more infection events, compared to the original data. Apart from these extreme cases, 12 out of
26 regions have an absolute, median deviation of less then 25 percent, 21 of the 26 regions show an absolute median below 50 percent
and 25 of the 26 regions have a median of less then 75 percent. Only the region ``Berstrasse'' had a higher deviation with
about -79.16 percent. Many of the plots are skewed right (\textcolor{red}{explain and describe better, also add data to Appendix!!!}).



%-----------------------------------
%	SUBSECTION 2
%-----------------------------------
\subsection{Simulating susceptibles in a 60 day time frame}
Similarly to the previous section, simulations were done done in a 60 day time frame. For this purpose only the first
60 days of the previously chosen time frame were simulated. \hyperref[fig:60_sim_expl]{Figure \ref*{fig:60_sim_expl}}
shows three exemplary graphs of the simulations. For better comparability the same regions were chosen as previously
in \hyperref[fig:76_sim_expl]{Figure \ref*{fig:76_sim_expl}}. The results were extrapolated in order to better visualize
the trend of the simulation result.

\begin{figure}
	\centering
	\begin{subfigure}[b]{0.3\textwidth}
		\centering
		\includegraphics[width=\textwidth]{./figures/60d/24_Darmstadt-Dieburg.png}	
		\caption{}
	\end{subfigure}
	\hfill
	\begin{subfigure}[b]{0.3\textwidth}
		\centering
		\includegraphics[width=\textwidth]{./figures/60d/13_Main-Kinzig-Kreis.png}	
		\caption{}
	\end{subfigure}
	\hfill
	\begin{subfigure}[b]{0.3\textwidth}
		\centering
		\includegraphics[width=\textwidth]{./figures/60d/10_Limburg-Weilburg.png}	
		\caption{}
	\end{subfigure}
	\caption{Three exemplary results of a simulation of the exposed individuals. The original (``orig'') data is drawn in blue,
		the simulated (``sim'') data is drawn in orange and the extrapolation of the simulated data (``extrapolation'') is
		drawn in green. The vertical red dotted line marks the transition from simulated to extrapolated data.
		}
	\label{fig:60_sim_expl}
\end{figure}

Figure \ref*{fig:60_sim_expl} shows that the general trend of the simulations remains similar to the simulations with 76 days.
In all three regions shown the extrapolated data points show a slightly slower increase in infections compared to the simulations
with 76 days in figure \ref*{fig:76_sim_expl}, but in all three cases the results are comparable.

Similar to the previous section, we calculated the deviation of each data point for each region and expressed the results in a
set of box plots. The results are shown in figure \ref*{fig:60_sim_box}.


\textcolor{red}{more text for box plots}



\begin{figure}
	\centering
	\begin{subfigure}[b]{0.4\textwidth}
		\centering
		\includegraphics[width=\textwidth]{./figures/60d/deviation_box60_alt1.png}	
	\end{subfigure}
	\begin{subfigure}[b]{0.4\textwidth}
		\centering
		\includegraphics[width=\textwidth]{./figures/60d/deviation_box60_alt2.png}	
	\end{subfigure}
	\begin{subfigure}[b]{0.4\textwidth}
		\centering
		\includegraphics[width=\textwidth]{./figures/60d/deviation_box60_alt3.png}	
	\end{subfigure}
	\begin{subfigure}[b]{0.4\textwidth}
		\centering
		\includegraphics[width=\textwidth]{./figures/60d/deviation_box60_alt4.png}	
	\end{subfigure}
	\caption{Shown are box plots of the percentage deviation, of every simulated region relative to the original data.}
	\label{fig:60_sim_box}
\end{figure}
%-----------------------------------
%	SUBSECTION 3
%-----------------------------------
\subsection{Simulating susceptibles in a 50 day time frame}

\textcolor{red}{still needs graphs and text for graphs and boxes}

\begin{figure}
	\centering
	\begin{subfigure}[b]{0.4\textwidth}
		\centering
		\includegraphics[width=\textwidth]{./figures/50d/deviation_box50_alt1.png}	
	\end{subfigure}
	\begin{subfigure}[b]{0.4\textwidth}
		\centering
		\includegraphics[width=\textwidth]{./figures/50d/deviation_box50_alt2.png}	
	\end{subfigure}
	\begin{subfigure}[b]{0.4\textwidth}
		\centering
		\includegraphics[width=\textwidth]{./figures/50d/deviation_box50_alt3.png}	
	\end{subfigure}
	\begin{subfigure}[b]{0.4\textwidth}
		\centering
		\includegraphics[width=\textwidth]{./figures/50d/deviation_box50_alt4.png}	
	\end{subfigure}
	\caption{Shown are box plots of the percentage deviation, of every simulated region relative to the original data.}
	\label{fig:50_sim_box}
\end{figure}
%----------------------------------------------------------------------------------------
%	SECTION 2
%----------------------------------------------------------------------------------------

\section{Simulation deviation in relation to region population}





%----------------------------------------------------------------------------------------
%	SECTION 3
%----------------------------------------------------------------------------------------

\section{Sensitivity analysis of $\alpha$ and $q$}

