% Chapter Template

\chapter{Introduction} % Main chapter title

\label{chap:introduction} % Change X to a consecutive number; for referencing this chapter elsewhere, use \ref{ChapterX}

%----------------------------------------------------------------------------------------
%	SECTION 1
%----------------------------------------------------------------------------------------

\section{Motivation}
Bacteria, viruses, parasites and all other kinds of other causes for diseases are omnipresent on our planet\cite{??}. Some are
relatively easy to overcome for most individuals, like the common cold\cite{??}. Others, such as HIV (human immunodeficiency virus),
can have severe long term impact on the quality of live of individuals\cite{??}. In addition to the impact sickness has
on the live of an individual, diseases and how they spread, have a huge impact on our societies. Hospitals need to be 
be build and staffed with doctors, nurses and other facility members. Supply chains need to be established and maintained in order
to manufacture drugs and materials that can be used to cure patients. Research has to be conducted and new treatments need to be
established, in order to increase the chances for a positive outcome of health complications. This entire network of services is
costly in terms of time, effort  and money used to both establish and maintain it.
%In 2019, the United States of America spend
%16.77\% of their GDP (gross domestic product) on healthcare expenditures\cite{WHO site}. This shows that managing public health
%is important for every society in order to both stay competitive economically and provide citizens with a high quality of live.\newline
\par
Even though huge progress has been made in the fight against infectious diseases (a notable example being the eradication of
small pox\cite{??}), sickness due to infectious diseases remain a common phenomenon in modern civilizations. Viruses are
among the most successful organisms in spreading between populations of both humans and animals\cite{??}. The most notable example
for this is the currently ongoing ``severe acute respiratory syndrome coronavirus 2'' (SARS-CoV-2) pandemic. The SARS-CoV-2 or
coronavirus disease 2019 (short COVID-19) was first observed in an outbreak in the Chinese city of Wuhan in December 2019\cite{??}.
From there it started to spread around the globe and on March, 11 2020, the world health organization declared a world wide pandemic\cite{??}.\newline
\par
Managing this pandemic has been a huge challenge in the past two years and much debate has revolved around managing the spread of this
infectious disease. To this end, scientists from many different fields have researched the properties of this virus in order to better
understand and predict how COVID-19 spreads in our societies. One of these fields is a branch of computer science and math, that aims to
model the spread of infectious diseases, such as  COVID-19, using differential equations. This work aims to contribute to this field and
%provide insight in the process of modeling the spread of infectious diseases on a two-dimensional grid using the SEIRD method. 
provide insight in the process of modeling the spread of infectious diseases, by building on the work of Rastogi et al.\cite{rastogi}.
In this work the ``SEIRD'' model\cite{wittum} was applied to seven major cities of Germany on a 2-dimensional (2D) grid. This work
follows up on this, by simulating COVID-19 spreading on 26 interconnected regions in the federal state of Hesse, Germany.


%----------------------------------------------------------------------------------------
%	SECTION 3
%----------------------------------------------------------------------------------------

%\section{Aims and goals}
%\textcolor{red}{Maybe move this more to the beginning? Difficult to formulate without previous explanations...}\newline % note
%\textcolor{red}{difficult to explain without model introduction}\newline % note
%This work aims to provide insight into the nature of the spreading of the COVID-19 virus. In a previous work Rastogi et al.\cite{rastogi}
%\textcolor{red}{(check for correct BA citation style)} % note
%explored the capability of the ``SEIRD'' model\cite{wittum} (detailed explanation in the \hyperref[sec:SEIRD]{following sections}).
%In his work he simulated the features of the COVID-19 pandemic in seven major cities of Germany on a 2-dimensional (2D) grid. While the
%results are interesting it remained unclear, how the model performs when simulating interconnected areas in a larger landscape.\newline
%
%\par
%In order to better understand the SEIRD model, a 2D grid of the German region of Hesse was created and used to simulate the spreading of
%the COVID-19 virus in the uninfected portion of the population. In addition, two parameters, that are regulating the spreading behavior
%in the model, were investigated closer. This analysis aimed to better understand the influence of the individual parameters on the modeling
%process itself.

