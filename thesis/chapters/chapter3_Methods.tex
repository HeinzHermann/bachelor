% Chapter Template

\chapter{Methodology} % Main chapter title
The following chapter will describe the problem that investigated in this work, the model adjustments and definitions,
the methods and the tools used to do so.

\label{chap:methodology} % Change X to a consecutive number; for referencing this chapter elsewhere, use \ref{ChapterX}

%----------------------------------------------------------------------------------------
%	SECTION 1
%----------------------------------------------------------------------------------------

\section{Problem definition}
\label{sec:problemDef}
%The goal of modeling is always to invent a model that is capable of correctly reproducing past and reliably predicting 
%future events. The latter is often very difficult, since it is necessary to correctly understand the dynamics of a
%complex system and then implement these dynamics in the form of a program, algorithm or method.\newline

%\par
The previous work of \cite{Rastogi} investigated the SEIRD model on a 2D-grid of Germany. In his work he simulated
all five groups of the SEIRD model for seven cities in Germany. This proved that the model itself is functional and
in principle capable of reproducing and predicting \textcolor{red}{(?)} the dynamic of the COVID-19 epidemic. This % note
is where this work continued.\newline

\par
The goal of this work was to better understand the performance of the model in a smaller scale environment and to improve
the quality of the results. Hesse (Germany) and its regions where chosen as a template for a new 2D-grid. To minimize the
propagation of error throughout the SEIRD groups, we decided to focus on optimizing the equation variables
\textcolor{red}{(add reference to equations)} % note
of each group successively. In this work we focused on the Susceptible (S) group and tried to reproduce/predict real world
data and upcoming trends. Additionally we focused on evaluating the sensitivity of the S group to the two variables
$\alpha$ and $q$.

%-----------------------------------
%	SECTION 2
%-----------------------------------
\section{Model adjustment and data acquisition}


%-----------------------------------
%	SUBSECTION 1
%-----------------------------------
\subsection{Data acquisition}
\label{sec:datacoll}
In order to run the SEIRD model, it was necessary to gather real world data for as many of the five groups as possible.
The source for this data was the Github project \cite{Gehrcke}, which provided both the number of COVID-19 infections,
as well as the number of COVID-19 related deaths per region in Germany. The repository provides data based on the case/death
numbers published by the RKI (Robert Koch Institute) and the Risklayer GmbH. In this work the data based on the RKI publications
was used.\newline

\par
The hospitalization rate of confirmed COVID-19 infected individuals, the average time till symptom onset, the average time till
hospitalization and the lethality rate case of an infection was taken from \cite{RKIcov}. At the time of acquisition, these
numbers were only estimations based on the literature of the current date.\newline

\par
The population of each region in Hesse was taken from the website ``statistic.hessen.de''. The data was sourced from \cite{HessePop}
in the category ``Bef\"olkerung in Hessen 2017 bis 2020 nach Verwaltungsbezirk in Monaten''. The size of each region in Hesse
was taken from \cite{HesseSize}.


\textcolor{red}{add covid19 variant}

%-----------------------------------
%	SUBSECTION 2
%-----------------------------------
\subsection{SEIRD model adjustments}
\label{sec:SEIRDredef}
In order to reach the goals described in \hyperref[sec:problemDef]{section \ref*{sec:problemDef}}, we decided to alter
the group definitions. We redefined the Exposed (E) and Infected (I) group as such:

\begin{enumerate}[label=$\bullet$]
	\item \B{Exposed (E)}: Group of individuals that are infected with the virus and will develop symptoms in the upcoming
		days. Individuals are infectious during this time and contribute to infections of susceptibles.
	\item \B{Infected (I)}: Group of individuals that has experienced symptoms. Individuals are mostly not infectious anymore
		and isolate themself/are isolated until recovery or death.
\end{enumerate}

Furthermore we decided to set the variable $\kappa$ to 1, effectively removing the option of exposed individuals to recover without
developing symptoms. These changes were done for two major reasons:\newline

\par
The first issue was data availability. The problem is, that the number of infected in the
collected data set only includes individuals that were tested positive with a polymerase chain reaction (PCR) test\cite{??}. However,
these tests are usually conducted, if an individual is experiencing symptoms of sickness\cite{??}. This means, that individuals, that do
not experience symptoms are (for the most part) not represented in the collected data. In addition, it is difficult to estimate
the ratio between infected individuals that experience symptoms and individuals that do not experience symptoms. This makes it
even harder establish realistic assumptions, on which bases the original data could be modified to account for this phenomenon.\newline

\par
The second reason for these changes is a better representation of the actual disease history of COVID-19. Generally individuals that
are infected with COVID-19 and are going to develop symptoms later, are infectious to others before they develop symptoms and only
remain infectious for a short time after symptom onset\cite{??}. In the meta-analysis \cite{casey2021presymptomatic}, Casey Bryards et al.
show that most of COVID-19 transitions are pre-symptomatic (the infection event occurs before symptom onset). Based on this idea, we
redefined the E and I groups in hopes of better modeling the real world dynamics of the transmission process.
\textcolor{red}{(needs better reasoning, since paper shows only about half of infections are pre-symptomatic)} %note


%----------------------------------------------------------------------------------------
%	SECTION 2
%----------------------------------------------------------------------------------------

\section{Assumptions and data pre-processing}
The following section will describe the assumptions made based on the previous explanations, the generation of a suitable data set and
the pre-processing of the newly created data.


%-----------------------------------
%	SUBSECTION 1
%-----------------------------------
\subsection{Assumptions}

