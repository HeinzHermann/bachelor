% Chapter Template

\chapter{Discussion} % Main chapter title
\label{chap:discussion} % Change X to a consecutive number; for referencing this chapter elsewhere, use \ref{ChapterX}

This chapter will summarize, discuss and interpret the results of the \hyperref[chap:results]{previous section
\ref*{chap:results}}. We will start by discussing and evaluating the accuracy of the model regarding the reproduction
of real world data, as the ability to predict future outcomes with fewer data points. After that, we will use the
analyze the results of the loss function analysis and the sensitivity analysis in order to better understand the results
and identify potential issues and areas of improvement. Lastly an outlook will be given for possible strategies and
ideas to improve both the implementation of the model and the model itself.
%----------------------------------------------------------------------------------------
%	SECTION 1
%----------------------------------------------------------------------------------------

\section{Simulation results compared to real life data}
\hyperref[sec:sim_res]{Section \ref*{sec:sim_sec}} has presented the simulation results produced by the current implementation
of the SEIRD model. It can be seen, that the accuracy of the simulation somewhat depends on the amount of data points supplied.
While the median susceptible deviation of all regions ranged between -80 and +140 percent, most the regions showed reasonable
amounts of deviation from the original data. In case of the 76 data point simulation, 21 of 26 regions showed an
absolute median deviation of 50 percent or less. Similar behavior was observed for the 60 and 50 data point simulations, in which
21 and 22 regions respectively showed an absolute median deviation of 50 percent or less. While the simulation accuracy of single regions
might differ between experiments (\textcolor{red}{double check this!}), results are overall comparable.

Predicting the original data proved more difficult. While the calculated trend for the 60 data point experiment was comparable
to the simulation with 76 data points, the 50 data point simulation showed strong differences and proved to be mostly inaccurate.\\
\textcolor{red}{needs hard data to support these claims, check python code on how to extract data points and add to results.
Then come back and redo this}\\ %note

A consistent observation throughout all experiments was a strong trend of the simulated regions to negatively deviate from the
original data. This means that the simulations generally predicted lower amounts of exposed individuals than seen in the original
data.



%-----------------------------------
%	SUBSECTION 1
%-----------------------------------
\subsection{Subsection 1}


%-----------------------------------
%	SUBSECTION 2
%-----------------------------------

\subsection{Subsection 2}

%----------------------------------------------------------------------------------------
%	SECTION 2
%----------------------------------------------------------------------------------------

\section{Section 2}

