% Chapter Template

\chapter{Discussion} % Main chapter title
\label{chap:discussion} % Change X to a consecutive number; for referencing this chapter elsewhere, use \ref{ChapterX}

This chapter will summarize, discuss and interpret the results of the \hyperref[chap:results]{previous section
\ref*{chap:results}}. We will start by discussing and evaluating the accuracy of the model regarding the reproduction
of real world data, as the ability to predict future outcomes with fewer data points. After that, we will use the
analyze the results of the loss function analysis and the sensitivity analysis in order to better understand the results
and identify potential issues and areas of improvement. Lastly an outlook will be given for possible strategies and
ideas to improve both the implementation of the model and the model itself.
%----------------------------------------------------------------------------------------
%	SECTION 1
%----------------------------------------------------------------------------------------

\section{Simulation results compared to real life data}
\hyperref[sec:sim_res]{Section \ref*{sec:sim_res}} has presented the simulation results produced by the current implementation
of the SEIRD model. It can be seen, that the accuracy of the simulation somewhat depends on the amount of data points supplied.
While the median susceptible deviation of all regions ranged between -80 and +140 percent, most the regions showed reasonable
amounts of deviation from the original data. In case of the 76 data point simulation, 21 of 26 regions showed an
absolute median deviation of 50 percent or less. Similar behavior was observed for the 60 and 50 data point simulations, in which
21 and 22 regions respectively showed an absolute median deviation of 50 percent or less. While the simulation accuracy of single regions
might differ between experiments (\textcolor{red}{double check this!}), results are overall comparable.\newline

Predicting the original data proved more difficult. While the calculated trend for the 60 data point experiment was comparable
to the simulation with 76 data points, the 50 data point simulation showed strong differences and proved to be mostly inaccurate.\\
\textcolor{red}{needs hard data to support these claims, check python code on how to extract data points and add to results.
Then come back and redo this}\\ %note

A consistent observation throughout all experiments was a strong trend of the simulated regions to negatively deviate from the
original data. This means that the simulations generally predicted lower amounts of exposed individuals than seen in the original
data. In all three experiments 19 of the 26 regions deviated in this way from the original data. This may seem odd at first,
since it could be expected, that the optimal state of the model is reached, when a maximum amount of regions are close to matching
their original data. However, if  closer attention if given to the way optimization was done in these experiments this phenomenon
becomes much clearer. All adjustments were done based on minimizing the loss function. The loss function itself was defined in
\hyperref[eq:loss_newton]{equation \ref*{eq:loss_newton}} as the sum of the square difference between the simulated and the original
data of all data points and each region. This means that a regions individual influence on the adjustment of model variables
increases as the total difference between its simulated and original data grows. This effect is further increased, since the
difference between original and simulated data is squared before it is summed up. Thus leading to a quadratic increase in influence
of highly deviating regions relative to the other regions. \newline

Combining this knowledge with the observations of the model results, helped us to identify three factors that can be used to explain
the optimization behavior of the current implementation of the model. \textcolor{red}{rephrase}.

\begin{enumerate}[label=\arabic*.]
	\item The percentage deviation of the simulated data from the original data.
	\item The time at which the deviation occurs.
	\item The total population of each individual region.
\end{enumerate}

The first factor is relatively obvious. The most influential region in our 76 data point experiment was ``Offenbach'' with
a mean deviation of -54 percent after optimization. This means, that about half of the simulated data points of this region
had less than half as many exposed individuals as the original data. Naturally this should cause adjustments of the model variables,
but it also transitions well to the second factor. The point in time when the deviation occurs has great influence on the 
overall influence of the region on the loss. In our experiments, many regions displayed strong deviations during the first
days of the simulation going up to 467.37 percent in the case of ``Limburg-Weilburg''. However, these differences were observed
during the first days of the simulation, where infection events in the real world were more sporadic, due to smaller numbers of
already infected individuals. This also means that the total difference between simulated and original data is usually much
smaller during the early days of the simulation, since the expected number of infection events at that point in time is much smaller.
In the case mentioned before, 467.37 percent deviation translated to a number of 4.6737 additional infection events in the simulation,
compared to only one infected person in the original data set. The influence on the overall loss was extremely low in this case.
\textcolor{red}{add a bit more information here like total loss for better comparability} %note
The third major factor is the overall population of each respective region, since regions with a high population will tend to
produce higher differences between simulated and original data compared to regions with smaller populations. This remains true, even
if the percentage deviation between simulated and original data is relatively small compared to other regions. A good example for this
is the region ``Frankfurt-am-Main'', which was the third strongest loss contributor in the optimized version of the model, despite
only showing a median deviation of about 21 percent between simulated and original data.

\textcolor{red}{add more info about farnkfurt and population; note that loss during optimization process was not reproduced.
Optimal variable state can only give hints to internal workings}

%-----------------------------------
%	SUBSECTION 1
%-----------------------------------
\subsection{Subsection 1}


%-----------------------------------
%	SUBSECTION 2
%-----------------------------------

\subsection{Subsection 2}

%----------------------------------------------------------------------------------------
%	SECTION 2
%----------------------------------------------------------------------------------------

\section{Section 2}

